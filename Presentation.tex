\documentclass[12pt,xcolor={dvipsnames}]{beamer}

%auto-ignore

%% Basic font packages
\newcommand{\hmmax}{0}
\newcommand{\bmmax}{0}
\usepackage[T3,T1]{fontenc}
%\DeclareMathAlphabet{\mathpzc}{OT1}{pzc}{m}{it} % For \mathpzc fonts
\usepackage{mathrsfs} % For \mathscr fonts
\usepackage{bm} % For \bm fonts
\makeatletter
\@ifclassloaded{beamer}
  { % if this is a beamer document, use sans-serif fonts
    \usepackage{tgheros}
    \renewcommand*\familydefault{\sfdefault}
  }
  { % otherwise:
    \ifx\asybeamer\@empty % use times for articles
    \usepackage[varg]{txfonts} % For math
    \usepackage{tgtermes} % For text
    \else % use beamer fonts for asy documents with beamer
    \usepackage{tgheros}
    \renewcommand*\familydefault{\sfdefault}
    \fi
  }
\usepackage{microtype} % For nicer alignment of text
\def\MT@register@subst@font{
  \MT@exp@one@n\MT@in@clist\font@name\MT@font@list
  \ifMT@inlist@\else\xdef\MT@font@list{\MT@font@list\font@name,}\fi}
\makeatother

% %% Compact the text (smaller intra-word spacing)
% \AtBeginDocument{
%   \setlength{\spaceskip}
%   {0.875\fontdimen2\font
%     plus 0.925\fontdimen3\font
%     minus 1\fontdimen4\font}
% }

%% Spread out the text lines slightly
\linespread{1.0425}

%% No extra space after periods
%\frenchspacing

%% A few necessary packages
\usepackage{graphicx} %
\usepackage{color} %
%\usepackage[all]{onlyamsmath} % Error on bad non-ams math
\usepackage{xspace}
%% And a few optional ones
\usepackage{braket}
\usepackage{accents}
%\usepackage{rotating} % 'sidewaystable' support
\usepackage{siunitx}
\usepackage{mathtools}
\DeclareSymbolFontAlphabet{\mathrm}{operators}

%% Define a few colors
\definecolor{CiteColor}{rgb}{0.18039, 0.18824, 0.57255}
%\definecolor{CiteColor}{rgb}{0.250, 0.250, 0.800}
\definecolor{UrlColor} {rgb}{0.741, 0.173, 0.000}
\definecolor{LinkColor}{rgb}{0.25098, 0.47843, 0.04706}
%\definecolor{LinkColor}{rgb}{0.000, 0.502, 0.118}


%%% Local Variables: 
%%% mode: latex
%%% TeX-master: "paper"
%%% End: 
 %
% \pagecolor{white}\usepackage{pdfpages} % For including other PDFs
\input{Externals/Macros} %
\input{Externals/Cornell} %
%\usepackage{multimedia} %
%\usepackage[setpagesize=false]{hyperref} %
%\usepackage{pgfpages} %
\usepackage[overlay]{textpos} %

\usepackage[noend]{algorithmic}
\usepackage{setspace}

\setlength{\TPHorizModule}{1cm}
\setlength{\TPVertModule}{1cm}

\renewcommand{\vec}{\bm}

\setbeamercolor{normal text}{fg=CornellDarkGray,%
  bg=CornellLightestGray}
\titlegraphic{\includegraphics[width=75pt, height=75pt]%
  {Externals/CULogoFineGray}}


%% For tikz/pgf plots
\usepackage{tikz} %
% \usepackage{tikzscale} %
\usepackage{pgfplots} %
\usepackage{pgfplotstable} %
\pgfplotsset{width=330pt} %
\pgfplotsset{compat=1.7} %
\usetikzlibrary{pgfplots.groupplots} %
\usetikzlibrary{calc} %
%\usetikzlibrary{decorations.shapes} %
%\usetikzlibrary{decorations.markings} %
%\usetikzlibrary{decorations.pathreplacing, calc} %
\pgfplotsset{axis line style={line width=1pt, black}} %
\pgfplotsset{tick style={line width=1pt, black}} %
\pgfplotsset{minor tick style={black}} %
\newcommand{\PlotData}{
  \addplot[color=Red, densely dotted, very thick]
  table[x index=0, y index=1] {\loadedtable};
  \addplot[color=Blue, dotted, very thick]
  table[x index=0, y index=2] {\loadedtable};
  \addplot[smooth, color=Green, very thick]
  table[x index=0, y index=3] {\loadedtable};
  \addplot[color=Blue]
  table[x index=0, y index=4] {\loadedtable};
  \addplot[color=Red]
  table[x index=0, y index=5] {\loadedtable};
}
%% Automatically write figures as PDFs for import next time
\usetikzlibrary{external}
\tikzexternalize
\tikzsetfigurename{Presentation_} %
%\tikzset{external/force remake} % To force reprocessing of plots

\renewcommand{\vec}[1]{\boldsymbol{#1}}

%\DeclareMathOperator*{\argmin}{arg\,min}

\graphicspath{{Plots/}{Plots/PenroseDiagrams}{Movies/}{Copies/}} %

\title{Waveform transformations}

\author[Mike Boyle] {Mike Boyle}

% \institute{American Physical Society}

\date{May 19, 2015}

\subject{}

\begin{document}

\begin{frame}[plain]
  \titlepage
\end{frame}


%%%%%%%%%%%%%%%%%%%%%%%%%%%%%%%%%%%%%%%%%%%%%%%%%%%%%%%%%%%%%%%%%%%%%%
%%%%%%%%%%%%%%%%%%%%%%%%%%%%%%%%%%%%%%%%%%%%%%%%%%%%%%%%%%%%%%%%%%%%%%

\begin{frame}
  \frametitle{Outline}
  \begin{itemize}
  \item Motivation
    \begin{itemize}
    \item Problems in the waveforms
    \item Center-of-mass drifts
    \item Cleaning up waveforms
    \end{itemize}
  \item Asymptotic symmetries (BMS group)
    \begin{itemize}
    \item Definition (with Penrose diagrams)
    \item Requirements
    \item Interpolation
    \item Spin-weighted functions
    \end{itemize}
  \item Conclusions
  \end{itemize}
\end{frame}


%%%%%%%%%%%%%%%%%%%%%%%%%%%%%%%%%%%%%%%%%%%%%%%%%%%%%%%%%%%%%%%%%%%%%%
%%%%%%%%%%%%%%%%%%%%%%%%%%%%%%%%%%%%%%%%%%%%%%%%%%%%%%%%%%%%%%%%%%%%%%

\part{Motivation}
\partpage


\begin{frame}
  \frametitle{A waveform mystery}
  SXS:BBH:0004
\end{frame}

\begin{frame}
  \frametitle{The center of mass}
  SXS:BBH:0004
\end{frame}

\begin{frame}
  \frametitle{Correcting the center of mass}
  \begin{equation*}
    \min_{\vec{x}_{0}, \vec{v}_{0}}\ \int_{t_{i}}^{t_{f}} \big\lvert
      \vec{x}_{\text{CoM}}(t) - (\vec{x}_{0} + \vec{v}_{0}\, t)
      \big\rvert^{2}\, d t
  \end{equation*}

  \pause
  \vspace{0.2in}
  \begin{equation*}
    \vec{x}_{0} = \frac{4 (t_{f}^{2} + t_{f} t_{i} + t_{i}^{2}) \int
      \vec{x}_{\text{CoM}}(t)\, d t - 6 (t_{f}+t_{i})\, \int
      \vec{x}_{\text{CoM}}(t)\, t\, d t} {(t_{f} - t_{i})^{3}}
  \end{equation*}

  \vspace{0.2in}
  \begin{equation*}
    \vec{v}_{0} = \frac{12 \int \vec{x}_{\text{CoM}}(t)\, t\, d t - 6
      \int \vec{x}_{\text{CoM}}(t)\, d t} {(t_{f} - t_{i})^{3}}
  \end{equation*}
\end{frame}

\begin{frame}
  \frametitle{Corrected center of mass}
  SXS:BBH:0004$'$
\end{frame}

\begin{frame}
  \frametitle{Corrected waveform}
  SXS:BBH:0004$'$
\end{frame}

\begin{frame}
  \frametitle{Correcting the catalog}
  \vspace{-0.2in}
  \begin{center}
    \includegraphics<1>[width=\linewidth]{DriftVelocity}
    \includegraphics<2>[width=\linewidth]{InitialDisplacement}
    \includegraphics<3>[width=\linewidth]{MaxDisplacement}
  \end{center}
\end{frame}

\begin{frame}
  \frametitle{Correcting the initial data}
  \begin{center}
    Ask Sergei!
  \end{center}
\end{frame}


%%%%%%%%%%%%%%%%%%%%%%%%%%%%%%%%%%%%%%%%%%%%%%%%%%%%%%%%%%%%%%%%%%%%%%
%%%%%%%%%%%%%%%%%%%%%%%%%%%%%%%%%%%%%%%%%%%%%%%%%%%%%%%%%%%%%%%%%%%%%%

\part{Asymptotic symmetries}
\partpage

\begin{frame}
  \frametitle{Standard ambiguities}
  \begin{itemize}
  \item Time translation
  \item<3-> Space translation
  \item Phase rotation
  \item<2-> General rotation
  \item<4-> Boost
  \end{itemize}

  \vspace{0.25in}

  \begin{center}
    \visible<5->{\Poincare group?}

    \visible<6>{Diffeomorphism group?!}
  \end{center}
\end{frame}

\begin{frame}
  \frametitle{Isolated system}
  Asymptotic flatness
  \begin{itemize}
  \item Radial coordinate $r$
  \item ``Regularity'' of manifold as $r \to \infty$
  \end{itemize}

  Bondi coordinates
  \begin{itemize}
  \item Null coordinate $u$
  \item Angular coordinates $x^{A} = (\theta, \phi)$
  \item Metric:
    \begin{multline*}
      ds^{2} = \frac{V}{r}e^{2\beta}\, du^{2} - 2e^{2\beta}\, du\, dr
      \\ + r^{2}\, h_{AB}\, (dx^{A} - U^{A}\, du)\, (dx^{B} - U^{B}\,
      du)
    \end{multline*}
  \end{itemize}
\end{frame}

\begin{frame}
  \frametitle{General symmetry group}
  \begin{itemize}
  \item Boost
  \item General rotation
  \item \alert{Supertranslation}
  \end{itemize}

  \vspace{0.25in}

  \begin{center}
    Bondi-Metzner-Sachs (BMS) group
  \end{center}
\end{frame}

\begin{frame}
  \frametitle{Coordinates on $\mathscr{I}^{+}$}
  \begin{columns}[T]
    \begin{column}{.5\textwidth}
      \only<1>{Penrose diagram} %
      \only<2>{Boosted observer} %
      \only<3>{Local time $\to$ retarded time \\ $u = t-r$} %
    \end{column}
    \hspace{-.15\textwidth}
    \begin{column}{.65\textwidth}
      \includegraphics<1>[width=\linewidth]{ObserverA}
      \includegraphics<2>[width=\linewidth]{ObserverB}
      \includegraphics<3>[width=\linewidth]{NullRays}
    \end{column}
  \end{columns}
\end{frame}

\begin{frame}
  \frametitle{Coordinates on $\mathscr{I}^{+}$}
  \begin{columns}[T]
    \hspace{.05\textwidth}
    \begin{column}{.5\textwidth}
      \only<1>{Sphere in $\mathcal{O}'$} %
      \only<2>{Points as seen in $\mathcal{O}$\\
        related by rotation} %
      \only<3>{Points as seen in $\mathcal{O}$\\
        related by boost} %
    \end{column}
    \hspace{-.1\textwidth}
    \begin{column}{.65\textwidth}
      \includegraphics<1>[width=0.8\linewidth]{OriginalGrid}
      \includegraphics<2>[width=0.8\linewidth]{DerotatedGrid}
      \includegraphics<3>[width=0.8\linewidth]{DeboostedGrid}
    \end{column}
  \end{columns}
\end{frame}

\begin{frame}
  \frametitle{Coordinates on $\mathscr{I}^{+}$}
  \begin{columns}[T]
    \begin{column}{.5\textwidth}
      \only<1>{Local time $\to$ retarded time \\ $u = t-r$} %
      \only<2>{Time translation \\
        $t \mapsto t+\delta t$ \\
        $u \mapsto u+\delta t$} %
      \only<3>{Space translation \\
        $\vec{x} \mapsto \vec{x} + \delta \vec{x}$ \\
        $u \mapsto u - \delta \vec{x} \cdot \hat{\vec{r}}$} %
      \only<4>{Spacetime translation \\
        $u \mapsto u+\sum_{\ell=0,1;m} \alpha_{\ell, m} Y_{\ell,
          m}(\theta, \phi)$} %
      \only<5->{Supertranslation \\
        $u \mapsto u+\sum_{\ell, m} \alpha_{\ell, m}
        Y_{\ell,m}(\theta, \phi)$} %
      % \only<11>{Supertranslation ambiguity} %
    \end{column}
    \hspace{-.15\textwidth}
    \begin{column}{.65\textwidth}
      \includegraphics<1>[width=\linewidth]{AsymptoticCoordinates}
      \includegraphics<2>[width=\linewidth]{TimeTranslation}
      \includegraphics<3>[width=\linewidth]{SpaceTranslation}
      \includegraphics<4>[width=\linewidth]{SpaceTimeTranslation}
      \includegraphics<5>[width=\linewidth]{SuperTranslation}
      \includegraphics<6>[width=\linewidth]{ExcludedRegion}
    \end{column}
  \end{columns}
\end{frame}

\begin{frame}
  \frametitle{BMS coordinate transformations}
  \begin{align*}
    \theta &\mapsto \theta' \\
    \phi &\mapsto \phi' \\
    u &\mapsto K(\theta', \phi') \left[u - \alpha(\theta', \phi')
        \right]
  \end{align*}

  \pause
  \begin{equation*}
    u' \approx \gamma\, \left[u - (\vec{x}_{0} + u\, \vec{v}_{0})
      \cdot \hat{\vec{r}} - \tilde{\alpha}(\theta', \phi') \right]
  \end{equation*}
\end{frame}

\begin{frame}
  \frametitle{BMS waveform transformations}
  \vspace{-0.2in}
  \begin{align*}
    r\, h %
    &\mapsto %
      \frac{e^{-2i \lambda}} {K} \left[ r\, h - \bar{\eth}^{2} \alpha
      \right] %
    \\
    r^{2}\, \sigma %
    &\mapsto %
      \frac{e^{2i \lambda}} {K} \left[ r^{2}\, \sigma - \eth^{2}
      \alpha \right] %
    \\
    c %
    &\mapsto %
      \frac{e^{2i \lambda}} {K} \left[ c - \eth^{2} \alpha \right] %
    \\[15pt]
    \text{News} \sim \frac{\partial c} {\partial u} %
    &\mapsto %
      \frac{e^{2i \lambda}} {K^{2}} \left[ \frac{\partial c}
      {\partial u} \right] %
    \\[15pt]
    r\, \Psi_{4} %
    &\mapsto %
      \frac{e^{-2i \lambda}} {K^{3}} \left[ r\, \Psi_{4}'\right]
  \end{align*}
\end{frame}

\begin{frame}
  \frametitle{BMS waveform transformations}
  \begin{equation*}
    \alt<2>{
      r\, h(u, \theta, \phi) \mapsto \frac{e^{-2i \lambda(\theta', \phi')}}
      {K(\theta', \phi')} \left[ r\, h(u', \theta', \phi') -
        \bar{\eth}^{2} \alpha(\theta', \phi') \right]}{r\, h \mapsto
      \frac{e^{-2i \lambda}} {K\vphantom{(0)}} \left[ r\, h -
        \bar{\eth}^{2} \alpha \right]}
  \end{equation*}
\end{frame}

% \begin{frame}
%   \frametitle{Effects on waveforms}
%   \only<1>{Simplified waveform (zero $\ell>2$)}
%   \only<2>{Rotation $10^{-3}$ about $\hat{x}$}
%   \only<3>{Time translation $\delta t = -1000$}
%   \only<4>{Space translation $\delta \vec{x} = (0.1, 0, 0)$}
%   \only<5>{Spacetime translation $\delta = (-1000, 0.1, 0, 0)$}
%   \only<6>{Boost $\vec{v} = (10^{-4}, 0, 0)$}
%   \only<7>{Supertranslation $\sim 10^{-2}$}
%   \begin{center}
%     \vspace{-0.15in}
%     \includegraphics<1>[width=\linewidth]{waveform}
%     \includegraphics<2>[width=\linewidth]{waveform_rotation}
%     \includegraphics<3>[width=\linewidth]{waveform_time_translation}
%     \includegraphics<4>[width=\linewidth]{waveform_space_translation}
%     \includegraphics<5>[width=\linewidth]
%       {waveform_spacetime_translation}
%     \includegraphics<6>[width=\linewidth]{waveform_boost}
%     \includegraphics<7>[width=\linewidth]{waveform_supertranslation}
%   \end{center}
% \end{frame}


\begin{frame}
  \frametitle{Spin-weighted functions (SWFs)}
  \begin{equation*}
    \Psi_{4}(\theta, \phi) = C_{abcd}(\theta, \phi)\, \bar{m}^{a}\,
    n^{b}\, \bar{m}^{c}\, n^{d}
  \end{equation*}

  \pause
  \begin{equation*}
    \bar{m}^{a} = \frac{1}{\sqrt{2}} ( \theta^{a} - i\, \phi^{a} )
  \end{equation*}

  \pause
  \vspace{0.2in}
  \begin{equation*}
    R: (\theta, \phi) \mapsto (\theta', \phi')
  \end{equation*}

  \begin{equation*}
    \Psi_{4}'(\theta', \phi') = \Psi_{4}(\theta, \phi)\,
    e^{-2i\lambda(\theta', \phi', R)}
  \end{equation*}
\end{frame}

\begin{frame}
  \frametitle{Spin-weighted functions (SWFs)}
  \begin{center}
    \includegraphics[width=0.6\linewidth]{DeboostedGrid}
  \end{center}
\end{frame}

\begin{frame}
  \frametitle{Spin-weighted functions (SWFs)}
  \begin{center}
    SWFs \emph{are not} functions on the sphere; \\[5pt]
    also need alignment of tangent space
  \end{center}
  
  \pause
  \vspace{0.3in}
  Might be tempted to think
  \begin{equation*}
    \Psi_{4} : S^{2} \times S^{1} \to \mathbb{C}
  \end{equation*}

  \pause
  Turns out
  \begin{equation*}
    \Psi_{4} : S^{3} \to \mathbb{C}
  \end{equation*}
  This is the Hopf bundle:
  \begin{equation*}
    S^{1} \hookrightarrow S^{3} \twoheadrightarrow S^{2}
  \end{equation*}
\end{frame}

\begin{frame}
  \frametitle{Spin-weighted functions (SWFs)}
  \begin{center}
    SWFs are functions on the spin group \\[10pt]
    Spin(3) = SU(2) = group of unit quaternions
  \end{center}
  \begin{columns}[T]
    \begin{column}{.3\textwidth}
      \begin{minipage}[c][.6\textheight][c]{1.8\textwidth}
        \begin{equation*}
          \hat{\vec{r}} = R\, \vec{z}\, R^{-1}
        \end{equation*}
      \end{minipage}
    \end{column}
    \begin{column}{.7\textwidth}
      \begin{gather*}
        \vec{l} = R\, \frac{\vec{t} + \vec{z}} {\sqrt{2}}\, R^{-1} \\
        \vec{n} = R\, \frac{\vec{t} - \vec{z}} {\sqrt{2}}\, R^{-1} \\
        \vec{m} = R\, \frac{\vec{x} + i\, \vec{y}} {\sqrt{2}}\, R^{-1}
        \\
        \bar{\vec{m}} = R\, \frac{\vec{x} - i\, \vec{y}} {\sqrt{2}}\,
        R^{-1}
      \end{gather*}
    \end{column}
  \end{columns}
\end{frame}

% \begin{frame}
%   \frametitle{Spin-weighted functions}
%   Weighted functions are contractions between tensors and tetrad
%   elements, as functions of position on the sphere.

%   Weighted functions are \emph{not} just functions on the sphere;
%   you need to specify a position on the sphere, as well as a choice
%   of the tetrad.

%   For spin-weighted functions, this is just an alignment of the
%   vectors in the tangent space --- equivalent to choice of unit
%   vector at each point.

%   But SW functions are not functions from $S^{2} \times S^{1}$.  We
%   can choose a section of that bundle.  But the Hairy-ball theorem
%   forbids continuous choice of such a thing for our purposes.

%   Instead, it's the Hopf bundle

%   SWSHs are really functions on the spin group; best to write them
%   as such
% \end{frame}

\begin{frame}
  \frametitle{BMS waveform transformations}
  \begin{equation*}
    \alt<2> %
    {r\, h(u, R) \mapsto \frac{1 \vphantom{e^{2}}} {K(R')} \left[ r\,
        h(u', R') - \bar{\eth}^{2} \alpha(R') \right]} %
    {r\, h(u, \theta, \phi) \mapsto \frac{e^{-2i \lambda(\theta', \phi')}}
      {K(\theta', \phi')} \left[ r\, h(u', \theta', \phi') -
        \bar{\eth}^{2} \alpha(\theta', \phi') \right]} %
  \end{equation*}
\end{frame}

\begin{frame}
  \frametitle{Rotor of a boost}
  \vspace{-0.3in}
  \begin{gather*}
    \Theta' = \arccos \left[ \hat{\vec{v}} \cdot
      \hat{\vec{r}}_{\theta', \phi'} \right]
    \\[15pt]
    \Theta = 2\, \arctan \left[ \sqrt{ \frac{1-\beta} {1+\beta} } \tan
      \frac{\Theta'}{2} \right]
    \\[15pt]
    R_{B} = \exp \left[ \frac{\Theta - \Theta'} {2}\, \frac{\vec{v}
        \times \vec{r}_{\theta', \phi'} } {\lvert \vec{v} \times
        \vec{r}_{\theta', \phi'} \rvert} \right]
    \\[20pt]
    \visible<2->{R' = R_{B}\, e^{\phi'\, \vec{z}/2}\, e^{\theta'\,
        \vec{y}/2}}
    % \\[15pt]
    % \visible<3->{ e^{-2i\lambda(\theta', \phi')}\, [r\, h(u',
    % \theta',
    % \phi') - \bar{\eth}^{2} \alpha(\theta', \phi')] = r\, h(u',
    % R_{\theta', \phi'}) - \bar{\eth}^{2}\alpha (R_{\theta', \phi'})}
  \end{gather*}
\end{frame}

% \begin{frame}
%   \frametitle{Implementing BMS transformations}
%   \begin{enumerate}
%   \item Evaluate
%     $\frac{r\, h(u_{i}, R_{\theta'_{j}, \phi'_{j}}) -
%       \bar{\eth}^{2}\alpha (R_{\theta'_{j}, \phi'_{j}})}
%     {K^{2}(\theta'_{j}, \phi'_{j})}$ for all $u_{i}$
%   \item Interpolate to $u'_{k}(\theta'_{j}, \phi'_{j})$
%   \item Repeat on each point $(\theta'_{j}, \phi'_{j})$ of
%     equiangular grid
%   \item Feed into \texttt{spinsfast} to get $h'^{\ell,m}(u'_{k})$
%   \item Repeat for each desired value of $u'_{k}$
%   \end{enumerate}
% \end{frame}

\begin{frame}
  \frametitle{Implementing BMS transformations}
  \vspace{-0.1in}
  \algsetup{indent=3em}
  \begin{spacing}{1.75}
    \begin{algorithmic}
      \FORALL{desired times slices $u'_{i}$}

      \FORALL{$(\theta'_{j}, \phi'_{j})$ on equiangular grid}

      \STATE
      $R'_{j} \gets R_{B}\, e^{\phi'_{j}\, \vec{z}/2}\, e^{\theta'_{j}\,
        \vec{y}/2}$

      \STATE
      $u_{i} \gets \frac{u'_{i}}{K(R')} + \alpha(R')$

      \FORALL{$u_{k}$ ``near'' $u_{i}$} %

      \STATE
      $r' h'(u_{k}, R'_{j}) \gets \frac{r\, h(u_{k}, R'_{j}) -
        \bar{\eth}^{2}\alpha (R'_{j})} {K(R'_{j})}$

      \ENDFOR

      \STATE
      $r'h'(u'_{i}, R'_{j}) \gets \texttt{interp} \big( [r' h'(u_{k},
      R'_{j})],\ u_{i} \big)$

      \ENDFOR

      \STATE
      $r'h'_{\ell, m}(u'_{i}) \gets \texttt{spinsfast} \big(
      [r'h'(u'_{i}, R'_{j})] \big)$

      \ENDFOR
    \end{algorithmic}
  \end{spacing}
\end{frame}

\begin{frame}
  \frametitle{Conclusions}
  \begin{itemize}
  \item No such thing as invariant waveform
  \item No preferred reference frame
  \item When comparing waveforms, frames must agree
  \end{itemize}

  \vspace{0.2in}
  Need BMS transformations for
  \begin{itemize}
  \item Accurate PN waveforms
  \item PN--NR comparisons
  \item Hybrid waveforms
  \item Making sense of ringdowns
  \end{itemize}
\end{frame}


\end{document}

%%% Local Variables:
%%% mode: latex
%%% TeX-master: t
%%% End:
